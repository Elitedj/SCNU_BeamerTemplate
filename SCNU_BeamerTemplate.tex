%# -*- coding:utf-8 -*-
\documentclass[10pt,aspectratio=43,mathserif]{beamer}		
%设置为 Beamer 文档类型,设置字体为 10pt,长宽比为4:3,数学字体为 serif 风格

%%%%-----导入宏包-----%%%%
\usepackage{scnu}			%导入 SCNU 模板宏包
\usepackage{ctex}			%导入 ctex 宏包,添加中文支持
\usepackage{amsmath,amsfonts,amssymb,bm}   %导入数学公式所需宏包
\usepackage{color}			 %字体颜色支持
\usepackage{graphicx,hyperref,url}
\usepackage{listings}
\usepackage{metalogo}	% 非必须
%% 上文引用的包可按实际情况自行增删
%%%%%%%%%%%%%%%%%%	


\beamertemplateballitem		%设置 Beamer 主题

%%%%------------------------%%%%%
\catcode`\。=\active         %或者=13
\newcommand{。}{.}				
%将正文中的“。”号转换为“.”。中文标点国家规范建议科技文献中的句号用圆点替代
%%%%%%%%%%%%%%%%%%%%%

%%%%----首页信息设置----%%%%
\title[SCNU Beamer Template]{SCNU Beamer Template}
\subtitle{————副标题}		
%%%%----标题设置


\author[Elitedj]{
  Elitedj \\\medskip
  {\small \url{10000@qq.com}}}
%%%%----个人信息设置

\institute[IOPP]{
  华南师范大学\\
  计算机学院
}
%%%%----机构信息

\date[Nov. 11 2020]{
  2020年11月11日}
%%%%----日期信息
  
\begin{document}

\begin{frame}
\titlepage
\end{frame}				%生成标题页


\section{}
\begin{frame}
\frametitle{目录}
\tableofcontents
\end{frame}				%生成提纲页

\section{无序列表}
\begin{frame}
\frametitle{无序列表测试}
\begin{itemize}
	\item {第一点}
	\item {第二点}
	\item {第三点}
	\begin{itemize}
		\item {第三点第一小点}
		\item {第三点第二小点}
	\end{itemize}
\end{itemize}
\end{frame}

\section{图片}
\begin{frame}
\frametitle{图片测试}
\begin{figure}[htbp]
	\centering
	\includegraphics[width=0.5\textwidth]{figures/my_photo.png}
	\caption{我的头像}\label{fig1}
\end{figure}
\end{frame}

\section{Q\&A}
\begin{frame}
\frametitle{Q\&A}
\begin{center}
	\textcolor[RGB]{0,45,84}{\Huge Q\&A}
\end{center}
\end{frame}

\section{Thanks}
\begin{frame}
\frametitle{\quad}
\begin{center}
	\textcolor[RGB]{0,45,84}{\Huge Thanks}
\end{center}
\end{frame}

\end{document}
